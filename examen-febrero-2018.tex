\documentclass[a4paper, 12pt]{article}

\usepackage[utf8]{inputenc}
\usepackage[T1]{fontenc}
\usepackage[spanish]{babel}
\usepackage{amsmath}
\usepackage{amssymb}
\usepackage{geometry}
\geometry{a4paper, left=20mm, top=18mm, bottom=18mm}

\newcommand{\R}{\mathbb R}


\begin{document}
\pagestyle {empty}
\begin{center}
\LARGE
Fundamentos matemáticos de la Arquitectura I

\Large
8 de febrero de 2018

\end{center}


%\large

MODELO A
\vskip 6mm
\begin{enumerate}
\item (1 pt)
Sea $v\in \R^2$ un vector, y $\alpha\in\R$ un escalar. 
Razonar si las dos siguientes afirmaciones son verdaderas o falsas:
$$
\| \alpha v \| = \alpha \| v \| \qquad \qquad \qquad
\| \alpha v\|^2 = \alpha^2 \| v\|^2
$$
\vskip 6mm

\item (2 pt)
Hallar el eje transversal de la hipérbola $xy-x-1=0$.
\vskip 6mm

\item
a) (0.5 pt) Sea $s(t)=(\cos t,\, -\sen t,\, 0)$.
Demostrar las siguientes igualdades:
$$
\|s\|=1, \qquad
\|s'\|=1, \qquad
s\cdot s'=0, \qquad
s''=-s, \qquad
\|s\times s'\|=1
$$
b) (0.5 pt) Hacer un dibujo aproximado de la curva $C$ parametrizada por
$$
r(t)=\frac{t}\pi\, s(t),\qquad  t\in(0,\infty).
$$
c) (1 pt)
Calcular la curvatura de $C$ en el punto $(-1,0,0)$ utilizando el apartado~a).

\hrulefill

\item (1.5 pt)
Hallar los extremos relativos de la divergencia del campo
$$
F(x,y)=(xy^2-8xy,\, x^2y+2xy)
$$
\vskip 6mm

\item (1.5 pt)
Uno de los siguientes campos vectoriales es conservativo. Identificarlo y hallar su potencial.
$$
F=(2xy^3z,\, 3xy^2z+z,\, x^2y^3+y+1),
\quad
G=(2xy^3z,\, 3x^2y^2z+z,\, x^2y^3+y+1)
$$
\vskip 6mm

\item (2 pt)
Hallar el volumen de la porción de la esfera unidad
$x^2 + y^2 +z^2 \le 1$
limitada inferiormente por el plano $z = 0$
y superiormente por el plano $z = 1/2$.
\vskip 6mm

\item (1.5 pt)
Cambiar el orden de integración (no es necesario resolver la integral).
$$
\int_1^2 \int_0^\frac{\ln y}4\,dx\,dy\, +
\int_2^{2\sqrt[3]{2}}\int_{\ln \frac{y}2}^\frac{\ln y}4\,dx\,dy
$$
\vskip 6mm


\item (1.5 pt)
Sea $k>0$ un número real.
Hallar la integral a lo largo de la recta que une los puntos $(1,1)$ y $(4,k)$ de la función
$$
f(x,y)=\frac{x(y-1)}{\sqrt{k^2-2k+10}}
$$
\vskip 6mm

\item (2 pt)
Utilizando el Teorema de Green, calcular la integral de línea sobre $C$ del campo vectorial $F(x,y)=(y,\, x^2+1)$, siendo $C$ la curva parametrizada por
$$
r(t)=(2\cos t, 2\sen t),\qquad t\in[0, 2\pi]
$$
\vskip 6mm

%\item
%Hallar el área de la porción del plano $x+2y+z=1$ limitada por los planos verticales $x=2$, $y=0$ y $x-y=1$.
%\vskip 6mm


\end{enumerate}

%\vskip 17mm
%\normalsize
%\noindent
%{\bf Puntuación:}
%\hskip 5mm
%{\bf 1:} 1 pt. \hskip 8mm 
%{\bf 2:} 2 pt. \hskip 8mm
%{\bf 3:} 2 pt. \hskip 8mm
%{\bf 4:} 1 pt. \hskip 8mm
%{\bf 5:} 2 pt. \hskip 8mm
%{\bf 6:} 2 pt.


\pagebreak
%%%%%%%%%%%%%%%%%%%%%%%%%%%%%%%%%%%%%%%%%%%%%%%%%%%%%%%%%%


\begin{center}
\LARGE
Fundamentos matemáticos de la Arquitectura I

\Large
8 de febrero de 2018

\end{center}

%\large

MODELO B
\vskip 6mm
\begin{enumerate}
\item (1 pt)
Sea $v\in \R^2$ un vector, y $\alpha\in\R$ un escalar. 
Razonar si las dos siguientes afirmaciones son verdaderas o falsas:
$$
\| \alpha v \| = \alpha \| v \| \qquad \qquad \qquad
\| \alpha v\|^2 = \alpha^2 \| v\|^2
$$
\vskip 6mm


\item (2 pt)
Hallar el eje transversal de la hipérbola $xy-y-1=0$.
\vskip 6mm

\item
a) (0.5 pt) Sea $s(t)=(\sen t,\, \cos t,\, 0)$.
Demostrar las siguientes igualdades:
$$
\|s\|=1, \qquad
\|s'\|=1, \qquad
s\cdot s'=0, \qquad
s''=-s, \qquad
\|s\times s'\|=1
$$
b) (0.5 pt) Hacer un dibujo aproximado de la curva $C$ parametrizada por
$$
r(t)=\frac{t}\pi\, s(t),\qquad  t\in(0,\infty).
$$
c) (1 pt)
Calcular la curvatura de $C$ en el punto $(0,-1,0)$ utilizando el apartado~a).

\hrulefill

\item (1.5 pt)
Hallar los extremos relativos de la divergencia del campo
$$
F(x,y)=(xy^2+4xy,\, x^2y-2xy)
$$
\vskip 6mm

\item (1.5 pt)
Uno de los siguientes campos vectoriales es conservativo. Identificarlo y hallar su potencial.
$$
F=(2xy^3z,\, 3x^2y^2z,\, x^2y^3+y+1),
\quad
G=(2xy^3z,\, 3x^2y^2z+z,\, x^2y^3+y+1)
$$
\vskip 6mm

\item (2 pt)
Hallar el volumen de la porción de la esfera unidad
$x^2 + y^2 +z^2 \le 1$
limitada inferiormente por el plano $z = 0$
y superiormente por el plano $z = 1/2$.
\vskip 6mm

\item (1.5 pt)
Cambiar el orden de integración (no es necesario resolver la integral).
$$
\int_1^3 \int_0^\frac{\ln y}3\,dx\,dy\, +
\int_3^{3\sqrt{3}}\int_{\ln \frac{y}3}^\frac{\ln y}3\,dx\,dy
$$
\vskip 6mm

\item (1.5 pt)
Sea $k>0$ un número real.
Hallar la integral a lo largo de la recta que une los puntos $(1,1)$ y $(k,3)$ de la función
$$
f(x,y)=\frac{(x-1)y}{\sqrt{k^2-2k+5}}
$$
\vskip 6mm

\item (2 pt)
Utilizando el Teorema de Green, calcular la integral de línea sobre $C$ del campo vectorial $F(x,y)=(y^2,\, x+1)$, siendo $C$ la curva parametrizada por
$$
r(t)=(2\cos t, 2\sen t),\qquad t\in[0, 2\pi]
$$
\vskip 6mm

%\item
%Hallar el área de la porción del plano $x+2y+z=1$ limitada por los planos verticales $x=2$, $y=0$ y $x-y=1$.
%\vskip 6mm

\end{enumerate}

%\vskip 17mm 
%\normalsize
%\noindent
%{\bf Puntuación:}
%\hskip 5mm
%{\bf 1:} 1 pt. \hskip 8mm 
%{\bf 2:} 2 pt. \hskip 8mm
%{\bf 3:} 2 pt. \hskip 8mm
%{\bf 4:} 1 pt. \hskip 8mm
%{\bf 5:} 2 pt. \hskip 8mm
%{\bf 6:} 2 pt.


\end{document}



