\documentclass[a4paper, 12pt]{article}

\usepackage[utf8]{inputenc}
\usepackage[T1]{fontenc}
\usepackage[spanish]{babel}
\usepackage{amsmath}
\usepackage{amssymb}
\usepackage{geometry}
\geometry{a4paper, left=20mm, top=20mm}

\newcommand{\R}{\mathbb R}


\begin{document}
\pagestyle {empty}
\begin{center}
\LARGE
Fundamentos matemáticos de la Arquitectura I

\Large
Control 1

\large
curso 2017-2018
\end{center}

\

\large

MODELO A
\vskip 5mm
\begin{enumerate}
\item
Sean $u,v,w\in \R^2$ tres vectores cualesquiera. Razonar si las siguientes afirmaciones son verdaderas o falsas
\begin{itemize}
\item[a)]
$u\cdot(v+w)=u\cdot v+ u\cdot w$
\item[b)]
$(u\cdot v)\,w=u\,(v\cdot w)$
\end{itemize}
\vskip 15mm


\item
Calcular las coordenadas cilíndricas y esféricas del punto $(1, 0, 1)$
\vskip 15mm
 
 
\item
Calcular el foco de la parábola
$$
x^2 + 2\sqrt{3} xy +3 y^2 + 8(\sqrt{3}-1)x -8(\sqrt{3}+1)y +16 = 0
$$
\vskip 15mm


\item
Hallar la ecuación canónica del conjunto de puntos cuya suma de distancias a los puntos $(0,0)$ y $(2,0)$ es $6$.
\vskip 15mm


\item
Calcular el centro de la siguiente cuádrica y decir de qué tipo es
$$
x^2+8y^2-z^2-2x-4z+5=0
$$
\vskip 15mm


\item Calcular el triedro de Frenet en el punto (1, -1, 1) de la curva
$$
r(t)=(t^2, -t, t^3)
$$

\end{enumerate}

\vskip 17mm
\normalsize
\noindent
{\bf Puntuación:}
\hskip 5mm
{\bf 1:} 2 pt. \hskip 8mm 
{\bf 2:} 1 pt. \hskip 8mm
{\bf 3:} 2 pt. \hskip 8mm
{\bf 4:} 2 pt. \hskip 8mm
{\bf 5:} 1 pt. \hskip 8mm
{\bf 6:} 2 pt.


\pagebreak
%%%%%%%%%%%%%%%%%%%%%%%%%%%%%%%%%%%%%%%%%%%%%%%%%%%%%%%%%%


\begin{center}
\LARGE
Fundamentos matemáticos de la Arquitectura I

\Large
Control 1

\large
curso 2017-2018
\end{center}

\

\large

MODELO B
\vskip 5mm
\begin{enumerate}
\item
Sean $u,v,w\in \R^2$ tres vectores cualesquiera. Razonar si las siguientes afirmaciones son verdaderas o falsas
\begin{itemize}
\item[a)]
$u\cdot(v+w)=u\cdot v+ u\cdot w$
\item[b)]
$(u\cdot v)\,w=u\,(v\cdot w)$
\end{itemize}
\vskip 15mm


\item
Calcular las coordenadas cilíndricas y esféricas del punto $(0,1, 1)$
\vskip 15mm
 

\item
Calcular el foco de la parábola
$$
3x^2 + 2\sqrt{3} xy + y^2 + 8(1-\sqrt{3})x -8(\sqrt{3}+1)y +16 = 0
$$
\vskip 15mm


\item
Hallar la ecuación canónica del conjunto de puntos cuya suma de distancias a los puntos $(0,0)$ y $(0,2)$ es $4$.
\vskip 15mm


\item
Calcular el centro de la siguiente cuádrica y decir de qué tipo es
$$
x^2+8y^2-4z^2+4x+8z-4=0
$$
\vskip 15mm


\item Calcular el triedro de Frenet en el punto (1, 1, -1) de la curva
$$
r(t)=(t, t^2, -t^3)
$$

\end{enumerate}

\vskip 17mm
\normalsize
\noindent
{\bf Puntuación:}
\hskip 5mm
{\bf 1:} 2 pt. \hskip 8mm 
{\bf 2:} 1 pt. \hskip 8mm
{\bf 3:} 2 pt. \hskip 8mm
{\bf 4:} 2 pt. \hskip 8mm
{\bf 5:} 1 pt. \hskip 8mm
{\bf 6:} 2 pt.
\pagebreak

\end{document}
