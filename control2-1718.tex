\documentclass[a4paper, 12pt]{article}

\usepackage[utf8]{inputenc}
\usepackage[T1]{fontenc}
\usepackage[spanish]{babel}
\usepackage{amsmath}
\usepackage{amssymb}
\usepackage{geometry}
\geometry{a4paper, left=20mm, top=20mm}

\newcommand{\R}{\mathbb R}


\begin{document}
\pagestyle {empty}
\begin{center}
\LARGE
Fundamentos matemáticos de la Arquitectura I

\Large
Control 2

\large
curso 2017-2018
\end{center}

\

\large

MODELO A
\vskip 5mm
\begin{enumerate}
\item
Calcular el {\bf área} del dominio de la función
$$
f(x,y)=\frac{\log(4-4x^2-y^2)}{\sqrt{1-(y+2)^2}}
$$
\vskip 25mm


\item
Determinar {\bf todos} los puntos críticos de la función
$
f(x,y)=x e^x \cos(y)
$
y decir cuáles de ellos son máximos relativos, mínimos relativos o puntos de silla.
\vskip 25mm



\item
Sea $k\in[0,1]$ un número real.
Hallar el volumen de la porción de la esfera unidad $x^2+y^2+z^2\le1$ limitada inferiormente por el plano $z=0$ y superiormente por el plano $z=k$.
\vskip 25mm


\item
Sea $S$ la intersección del cilindro $y^2+z^2\le1$ y el plano $x=1$. Calcular la integral de superfice sobre $S$ del campo vectorial
$$
F(x,y,z)=(x^2, \, \arctan (x+y), \, x^4+\sen z \cos y)
$$
\vskip 25mm
\end{enumerate}

\vfill
\normalsize
\noindent
{\bf Puntuación:}
\hskip 10mm
{\bf 1:} 3 pt. \hskip 16mm 
{\bf 2:} 2 pt. \hskip 16mm
{\bf 3:} 3 pt. \hskip 16mm
{\bf 4:} 2 pt.


\pagebreak
%%%%%%%%%%%%%%%%%%%%%%%%%%%%%%%%%%%%%%%%%%%%%%%%%%%%%%%%%%


\begin{center}
\LARGE
Fundamentos matemáticos de la Arquitectura I

\Large
Control 2

\large
curso 2017-2018
\end{center}

\

\large

MODELO B
\vskip 5mm
\begin{enumerate}
\item
Calcular el {\bf área} del dominio de la función
$$
f(x,y)=\frac{\log(4-x^2-4y^2)}{\sqrt{1-(x+2)^2}}
$$
\vskip 25mm

\item
Determinar {\bf todos} los puntos críticos de la función
$
f(x,y)=y e^y \cos(x)
$
y decir cuáles de ellos son máximos relativos, mínimos relativos o puntos de silla.
\vskip 25mm
 
 
\item
Sea $k\in[0,1]$ un número real.
Hallar el volumen de la porción de la esfera unidad $x^2+y^2+z^2\le1$ limitada inferiormente por el plano $z=0$ y superiormente por el plano $z=k$.
\vskip 25mm


\item
Sea $S$ la intersección del cilindro $x^2+z^2\le1$ y el plano $y=1$. Calcular la integral de superfice sobre $S$ del campo vectorial
$$
F(x,y,z)=(\arctan(x-y), \, y^2, \, x^3+\cos z \sen x)
$$
\vskip 25mm

\end{enumerate}

\vfill
\normalsize
\noindent
{\bf Puntuación:}
\hskip 10mm
{\bf 1:} 3 pt. \hskip 16mm 
{\bf 2:} 2 pt. \hskip 16mm
{\bf 3:} 3 pt. \hskip 16mm
{\bf 4:} 2 pt.

\end{document}
